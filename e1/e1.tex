\documentclass[10pt,twoside,twocolumn,openany]{dndbook}

\usepackage[swedish]{babel}
\usepackage[utf8]{inputenc}
\usepackage[hidelinks]{hyperref}

\title{Midvinternatt \\ \large{\textit{Fornnordiskt rollspel}}}
\author{Lapsang Boys}
\date{25 januari 2020}

\begin{document}

\frontmatter

\maketitle

\tableofcontents

\mainmatter

% === [ Äventyr ] ==============================================================

\chapter{Äventyr}

\DndDropCapLine{M}{idvinternatt är ett fornnordiskt rollspel} som tar inspiration från \emph{Dungeons and Dragons}, \emph{Eon} och \emph{Mutant Chronicles}.

% === [ Väsen ] ================================================================

\chapter{Väsen}

\DndDropCapLine{V}{äsen och folktro.}

\vspace{4em}

% --- [ Askefroa ] -------------------------------------------------------------

\section{Askefroa}

Se \ref{fig:folktro_askefroa}.

\begin{figure*}[htbp]
	\center
	\includegraphics[width=\textwidth]{inc/folktro/f029-askefroa.jpg}
	\caption{Askefroa}
	\label{fig:folktro_askefroa}
\end{figure*}

% --- [ Havsfru ] --------------------------------------------------------------

\section{Havsfru}

Se \ref{fig:folktro_havsfru}.

\begin{figure*}[htbp]
	\center
	\includegraphics[width=\textwidth]{inc/folktro/f036-havsfru.jpg}
	\caption{Havsfru}
	\label{fig:folktro_havsfru}
\end{figure*}

% --- [ Källrå ] --------------------------------------------------------------

\section{Källrå}

Se \ref{fig:folktro_källrå}.

\begin{figure*}[htbp]
	\center
	\includegraphics[width=\textwidth]{inc/folktro/f045-källrå.jpg}
	\caption{Källrå}
	\label{fig:folktro_källrå}
\end{figure*}

% --- [ Vättar ] --------------------------------------------------------------

\section{Vättar}

Se \ref{fig:folktro_smågabba}, \ref{fig:folktro_vättekvinna}, \ref{fig:folktro_vätte} och \ref{fig:folktro_vätteryttare}.

\begin{figure*}[htbp]
	\center
	\includegraphics[width=\textwidth]{inc/folktro/f071-smågabba.jpg}
	\caption{Smågabba}
	\label{fig:folktro_smågabba}
\end{figure*}

\begin{figure*}[htbp]
	\center
	\includegraphics[width=\textwidth]{inc/folktro/f072-vättekvinna.jpg}
	\caption{Vättekvinna}
	\label{fig:folktro_vättekvinna}
\end{figure*}

\begin{figure*}[htbp]
	\center
	\includegraphics[width=\textwidth]{inc/folktro/f073-vätte.jpg}
	\caption{Vätte}
	\label{fig:folktro_vätte}
\end{figure*}

\begin{figure*}[htbp]
	\center
	\includegraphics[width=0.9\textwidth]{inc/folktro/f074-vätteryttare.jpg}
	\caption{Vätteryttare}
	\label{fig:folktro_vätteryttare}
\end{figure*}

% === [ Slutord ] ==============================================================

\onecolumn

\chapter{Slutord från författarna}

\DndDropCapLine{K}{eep it lapsang}

\vspace{6em}

\begin{figure}[htbp]
	\center
	\includegraphics[width=0.7\textwidth]{inc/folktro/f124-uggla.jpg}
\end{figure}

\textit{Illustrationer från ``Nordiska Väsen'' och ``Nordiska Gudar'' är skapade av Johan Egerkrans.}

\end{document}
